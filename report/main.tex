\documentclass{article}

\usepackage{graphicx}
\usepackage{algorithm}
\usepackage{amsmath}
\usepackage{float}
\usepackage{subcaption}
\usepackage{hyperref}

\title{Parallel Huffman}
\author{Giulio Piva}
\date{\today}

\begin{document}

\maketitle

\tableofcontents
\break
\section{Problem statement}
The Huffman algorithm is a lossless data compression algorithm that employs a variable-length
code table for encoding a sequence. This code table is derived from the probability of
occurrence of each distinct symbol in the input. In this scheme, the most frequently
occurring symbol is encoded with the shortest code, while the least frequent symbol receives the
longest code. The Huffman algorithm can be broadly divided into four major steps:
\begin{enumerate}
    \item Frequency counting: The first operation to be performed is generating the frequency map of the characters in the input sequence.
          The time complexity of this operation is $O(n)$, where $n$ is the length of the input sequence.
    \item Huffman Tree generation: The tree is built by iteratively picking the two nodes with the lowest frequency and connect them to a parent node.
          In this way, the characters with higher frequency are matched with a node closer to the root of the tree and consequently
          in the encoding it will result in shorter code. The most appropriate data structure to
          support this step is a priority queue, which allows to retrieve the nodes with the lowest
          frequency in $O(1)$ time and to insert a new node in $O(log(m))$ time.
          Therefore, the time complexity of this step is $O(mlog(m))$, where $m$ is the number of different characters in the input sequence. Since
          we are dealing with ASCII characters, $m$ is at most 256 and therefore the time complexity is constant.
    \item Code generation: This phase saves the huffman codes produced by the previous step in a hashmap. It
          is done to avoid traversing the Huffman tree for every character in the input sequence and providing fast access to the codes.
          The time complexity of this step is $O(m)$.
    \item Encoding: The last step is the actual encoding of the input sequence. This is done by traversing the input sequence and
          for every character, appending the corresponding code to the output sequence. Again The time complexity of this step is $O(n)$.
\end{enumerate}
We must also consider the time of the the operations for reading and writing the file.
As we can see already from the time complexities, the most expensive operations are the frequency counting and encoding steps.
For large input sequences, we can expect them to become primary sources of bottlenecks, together with the I/O operations.
In the following sections, I will describe the parallelization strategies adopted to overcome these bottlenecks.

\section{Implementation}
This project employs the template design pattern. The primary class, \textbf{huffman\_base}, outlines the algorithm's structure.
This base class specifies the sequence of function executions, where the parallel functions are defined as pure virtual functions.
It also incorporates boilerplate code for file reading and writing (binary) files, thereby eliminating code duplication.
For each variant of the algorithm, a distinct class has been created: \textbf{huffman\_sequential}, \textbf{huffman\_thread}, and \textbf{huffman\_ff}.
This design pattern aligns naturally with the project's requirements, as the same algorithm needs to be implemented in different ways.
Furthermore, it provides a straightforward method for implementing new (parallel) versions of the algorithm.
The functions to be specialized are:
\begin{itemize}
    \item \begin{verbatim}
virtual encoded_data* encode_string(
      std::unordered_map<char, std::vector<bool>*>& codes,
      std::string &text) = 0;
\end{verbatim}
    \item \begin{verbatim}
virtual std::unordered_map<char, unsigned int> count_frequency(
      std::string &text) = 0;
    \end{verbatim}
\end{itemize}

\subsection{Data structures}
\subsubsection{Table codes}
The codes of the characters are stored in a hashmap of type
\begin{verbatim}
std::unordered_map<char, std::vector<bool>*>
\end{verbatim}
The choice of a vector of bools to represent the codes is dictated by its space efficiency, since it uses a single bit for each element
as opposed to a char which uses 8 bits for each one.
The purpose of the pointers instead is to avoid unnecessary data copying or movement during the encoding phase.
\subsubsection{Encoded sequence}
For storing the encoded sequence I used a data structure with the following type

\begin{verbatim}
typedef std::vector<std::vector<std::vector<bool>*>*> encoded_data;
\end{verbatim}

The inner vector represents a
pointer to the encoding of a character. The middle vector
represents a part of the input sequence encoded by a single worker.
Finally, the outer vector holds a collection of these chunks which represent
the whole encoded sequence.
One advantage of this data structure is to avoid the reduce phase.
During the write phase, all the produced codes must be traversed one by one, regardless
of them being concatenated in a single vector or split into chunks. In such a manner we
avoid the non-negligible overhead of concatenating the subsequences.
The sequential version implementation doesn't not require particular adaptions as it
will produce a vector composed of a single chunk (the whole sequence encoded).
This datastructure provides also advantages in terms of memory access.
In a NUMA system (the reference architecture for this project), each processor (or core) has its own local memory,
and accessing this local memory is faster than accessing memory that is local to another processor.
Having a small vector also enhances the chances of the data being stored in the some level of cache.

\subsection{Thread version}
\subsubsection{Frequency counting}
To parallelize the frequency counting, I adopted a standard
map-reduce pattern. Every thread is assigned statically a portion of
the input sequence to compute a partial frequency map.
When all the threads have finished their task and are joined together,
the produced partial frequency maps are merged together to obtain the final
frequency map. The reduce phase is not parallelized:
In our particular scenario, given that we are working with a maximum of 256 characters (ASCII),
this phase entails the summation of up to 256 integers from each mapper.
This operation is relatively lightweight and the benefits of parallelization are negligible
or might even be detrimental.
\subsubsection{Encoding phase}
Building upon the previously described data structure, this phase employs a standard map pattern.
The input sequence is statically divided into chunks, with each thread independently processing a
chunk to produce encoded subsequences.
Each worker identifies its index and calculates the encoding for every character within
its assigned segment of the sequence. This is accomplished by appending the corresponding
character's pointer to the appropriate chunk. Once this process is complete, the worker returns
the encoded sequence to the invoking thread.


\subsection{FastFlow version}
The FastFlow implementation is similar to the previous one and
implements the same patterns described before, in an high level manner.
\subsubsection{Frequency counting}
FastFlow provides naively high level data parallel patterns. It was simple
to implement the parallel-reduce function taking advantage of the ff::ParallelForReduce
class and the parallel\_reduce method.
Specifically, two primary lambda functions were defined: one for mapping and another for reducing.
The mapping lambda function is responsible for incrementing the count for the character under examination.
The reducing function, on the other hand, handles the merging of results.
\subsubsection{Encoding phase}
The implementation of this phase utilizes the ParallelFor construct of FastFlow.
This approach was chosen both for its simplicity and have a sort of fair comparison
with the threaded version. The body of the parallel for is essentially identical
to that in the threaded version.
An alternative implementation could have utilized the \textbf{ff\_Farm} pattern.
This would have necessitated the definition of an Emitter as an instance of \textbf{ff\_monode\ t}.
This instance would create an object of type \textbf{Task} containing details about the portion of the sequence to be
encoded by a worker. Subsequently, A \textbf{Collector} object would collect the chunks produced by the workers into the final result.
This approach requires a certain amount more of code to be written. Moreover, The ParallelFor
construct is based on the farm building block, therefore the performances should be comparable.

\section{Benchmark}
This section presents the results from the conducted tests. The tests were executed on a dual-socket NUMA
AMD EPYC 7301 machine, equipped with 16 cores and 32 threads each, totaling 64 hardware threads.
For each thread count, the execution time was measured five times, and the average time was calculated.
Each version of the Huffman algorithm was tested in conjunction with the \textbf{jemalloc} library.
The application is compiled with the -O3 flag.
\subsection{50kb file}
Theoretically, the speedup on such a lightweight file should be very low.
The obtained results depicted in figure \ref{fig:50kb} in fact confirm this hypothesis: The overhead derived
from the setup of the parallelization overcomes the sequential execution time.
With the thread version, we can observe that only up to 2 threads a small speedup is achieved ($\approx 1.2$).
Then, the execution time increases as the number of threads increases.
With FastFlow there is not even a speedup with this file. The introduced overheads
for setting up the parallelization and moving data around leads to a performance decrease
with respect to the sequential version.
\begin{figure}[H]
    \begin{subfigure}{0.6\textwidth}
        \centering
        \includegraphics[width=1\textwidth]{figures/64k/speedup.png}
        \caption{speedup with 50 kb file (no I/O)}
    \end{subfigure}
    \begin{subfigure}{0.6\textwidth}
        \centering
        \includegraphics[width=1\textwidth]{figures/64k/efficiency.png}
        \caption{efficiency with 50 kb file (no I/O)}
    \end{subfigure}
    \begin{subfigure}{0.6\textwidth}
        \centering
        \includegraphics[width=1\textwidth]{figures/64k/total-time.png}
        \caption{execution time ($\mu s$) with 50 kb file (no I/O)}
    \end{subfigure}
    \caption{Metrics for 50 kb file}
    \label{fig:50kb}
\end{figure}

\subsection{50MB file}
With a reasonable large file size, the benefits provided by the parallelization emerge.
We observe in figure \ref{fig:50mb} that the threads version reaches a peak speedup of $\approx 15$ with 64 threads,
whereas the FastFlow version obtains a speedup of $\approx 12$ with 32 threads.
The observed smaller speedup might be a result of FastFlow's communication channels managing
increasingly granular computations as the number of workers escalates.
Moreover, this graph has been produced with the flag \textbf{NO\_DEFAULT\_MAPPING}
compile option, which
disables the default thread mapping provided by FastFlow and put the OS in charge to perform it.
With this option, it might be the case that the FastFlow nodes are moved from one core to another
during the execution and this is a additional overhead since a context switch in envolved.
We can observe that the jemalloc library doesn't provide concrete benefits
and starting from a certain number of threads it even worsen the performance.
As forecasted, the I/O operations turned out to be very expensive and if
we consider the total time accounting for them, the speedup drops significantly.
Example metrics are shown in table \ref{tab:measurements}.
\begin{table}[H]
    \centering
    \begin{tabular}{|c|c|c|c|c|}
        \hline
        Implementation & Workers & Time (I/0) & Time (no I/O) \\
        \hline
        Sequential     & -       & 2748534    & 1645368       \\
        Threads        & 64      & 1192910    & 99086         \\
        FastFlow       & 32      & 1207970    & 131287        \\
        \hline
    \end{tabular}
    \caption{Example measurements for 50mb file}
    \label{tab:measurements}
\end{table}

\begin{figure}
    \begin{subfigure}{0.6\textwidth}
        \centering
        \includegraphics[width=1\textwidth]{figures/64M/speedup.png}
        \caption{speedup with 50 MB file (no I/O)}
        \label{fig:speedup-50mb}
    \end{subfigure}
    \begin{subfigure}{0.6\textwidth}
        \centering
        \includegraphics[width=1\textwidth]{figures/64M/efficiency.png}
        \caption{efficiency with 50 MB file (no I/O)}
        \label{fig:efficiency-50mb}
    \end{subfigure}
    \begin{subfigure}{0.6\textwidth}
        \centering
        \includegraphics[width=1\textwidth]{figures/64M/total-time.png}
        \caption{execution time ($\mu s$) with 50 MB file (no I/O)}
        \label{fig:total-time-50mb}
    \end{subfigure}
    \caption{Metrics for 50 MB file}
    \label{fig:50mb}
\end{figure}

\subsection{50MB file - default thread mapping}
The performance tests has been conducted also using the default thread mapping provided by FastFlow.
We observe in figure  \hyperref[fig:speedup-nomap]{Figure \ref*{fig:speedup-nomap}} that the FastFlow implementation experiences a significant decrease in speedup beyond 16 threads.
This suggests that the default thread mapping may not be optimally configured for NUMA architecture,
leading to inefficient memory access patterns and subsequently to a degradation in performance.
NUMA architectures have multiple memory nodes with varying access speeds depending on the proximity to the processing unit.
The observed performance drop could be due to thread contention or memory bandwidth saturation when multiple threads access
non-local memory simultaneously.
Notably, the integration of the jemalloc memory allocator yields a substantial improvement,
with the speedup exceeding that of the standard threaded version.
The enhanced performance is likely attributable to jemalloc's NUMA-awareness,
which enables it to allocate memory closer to the executing threads, reducing latency and increasing memory access speed.

\begin{figure}
    \begin{subfigure}{0.6\textwidth}
        \centering
        \includegraphics[width=1\textwidth]{figures/64M-nomap/speedup.png}
        \caption{speedup with 50 MB file (no I/O)}
        \label{fig:speedup-nomap}
    \end{subfigure}
    \begin{subfigure}{0.6\textwidth}
        \centering
        \includegraphics[width=1\textwidth]{figures/64M-nomap/efficiency.png}
        \caption{efficiency with 50 MB file (no I/O)}
        \label{fig:efficiency-nomap}
    \end{subfigure}
    \begin {subfigure}{0.6\textwidth}
        \centering
        \includegraphics[width=1\textwidth]{figures/64M-nomap/total-time.png}
        \caption{execution time ($\mu s$) with 50 MB file (no I/O)}
        \label{fig:total-time-nomap}
    \end{subfigure}
    \caption{Metrics for 50 MB file with default thread mapping}
    \label{fig:50mb-nomap}
\end{figure}

\section {Conclusions}
This project has provided a comprehensive overview of the various aspects involved in parallel programming.
Memory management plays a crucial role in the performance of a parallel application, and understanding the
underlying architecture of the machine executing the application is equally important. The results obtained
confirm that I/O operations can significantly impact the performance of Huffman encoding. Future work could
focus on enhancing the performance of these I/O operations. However, it's important to note that their performance
is also largely dependent on the underlying storage system.

\section{Compilation}
It is possible to compile the project by running the following commands:
\begin{verbatim}
cmake .
make
\end{verbatim}
and then the program can be used as follows:
\begin{verbatim}
./HuffmanProject <input_file> <output_file> <seq|t|ff> <n_threads>
\end{verbatim}
The script for running the benchmarks is \textbf{measurements.sh}.
It can be executed as follows:
\begin{verbatim}
    ./measurements.sh <file> <jemalloc\_path>
\end{verbatim}
the benchmarks are saved in the \textbf{measurements} folder.
If you want to test the correctness of the encoding you should uncomment the \textit{add\_definitions(-DDECODE)} option in the CMakeLists.txt file
and recompile the project.
The Plot script is \textbf{plot.py} and can be executed as follows:
\begin{verbatim}
    python3 plot.py
\end{verbatim}
The resulting figures are saved in the \textbf{figures} folder.
You should install the matplotlib, numpy and pandas libraries to run the script.
\end{document}